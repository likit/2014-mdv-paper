% Template for PLoS
% Version 1.0 January 2009
%
% To compile to pdf, run:
% latex plos.template
% bibtex plos.template
% latex plos.template
% latex plos.template
% dvipdf plos.template

\documentclass[10pt]{article}

% amsmath package, useful for mathematical formulas
\usepackage{amsmath}
% amssymb package, useful for mathematical symbols
\usepackage{amssymb}

% graphicx package, useful for including eps and pdf graphics
% include graphics with the command \includegraphics
\usepackage{graphicx}

% cite package, to clean up citations in the main text. Do not remove.
\usepackage{cite}

\usepackage{color} 

% Use doublespacing - comment out for single spacing
%\usepackage{setspace} 
%\doublespacing


% Text layout
\topmargin 0.0cm
\oddsidemargin 0.5cm
\evensidemargin 0.5cm
\textwidth 16cm 
\textheight 21cm

% Bold the 'Figure #' in the caption and separate it with a period
% Captions will be left justified
\usepackage[labelfont=bf,labelsep=period,justification=raggedright]{caption}

% Use the PLoS provided bibtex style
\bibliographystyle{plos2009}

% Remove brackets from numbering in List of References
\makeatletter
\renewcommand{\@biblabel}[1]{\quad#1.}
\makeatother


% Leave date blank
\date{}

\pagestyle{myheadings}
%% ** EDIT HERE **


%% ** EDIT HERE **
%% PLEASE INCLUDE ALL MACROS BELOW

%% END MACROS SECTION

\begin{document}

% Title must be 150 characters or less
\begin{flushleft}
{\Large
\textbf{Transcriptome variation in response to Marek's disease virus early infection}
}
% Insert Author names, affiliations and corresponding author email.
\\
Likit Preeyanon$^{1}$
C. Titus Brown$^{1,2}$
Hans H. Cheng$^{3,\ast}$
\\
\bf{1} Microbiology and Molecular Genetics, Michigan State University, East Lansing, MI, USA
\\
\bf{2} Computer Science and Engineering, Michigan State University, East Lansing, MI, USA
\\
\bf{3} USDA, ARS, Avian Disease and Oncology Laboratory, East Lansing, MI, USA
\\
$\ast$ E-mail: Corresponding hans.cheng@ars.usda.gov
\end{flushleft}

% Please keep the abstract between 250 and 300 words
\section*{Abstract}
Marek's disease (MD) is caused by highly oncogenic Marek's
disease virus (MDV).  Understanding the genetic basis for non-MHC
resistance would be of both fundamental and applied importance.
In this study, using computational approaches of RNA-Seq data, we
identified differentially expressed genes and isoforms in
response to MDV infection from highly inbred chickens lines that
are MD resistant (line 6), susceptible (line 7), and performed
pathway and functional analysis.  Results show that expression of
genes and isoforms involved in cell cytoskeleton and adhesion
molecules in the resistant line may contribute to resistance of
the disease by limiting T cell activation, which in turn prevents
the spread of MDV to activated T cells.  In addition, we
identified exonic single nucleotide polymorphisms (SNPs) between
lines predicted role in regulation of alternative splicing.
Based on our results, we speculate that alternative splice forms
of cell adhesion molecules and cytoskeleton play and important
role in controlling T cell activation in the resistant line.

% Please keep the Author Summary between 150 and 200 words
% Use first person. PLoS ONE authors please skip this step. 
% Author Summary not valid for PLoS ONE submissions.   
\section*{Author Summary}

\section*{Introduction}

Marek's disease (MD) is an economically significant chicken
disease that affects the poultry industry worldwide with
estimated annual of \$2 billion~\cite{morrow2004marek}.  The
disease is caused by the highly oncogenic Marek's disease virus
(MDV), an alphaherpesvirus that induces T-cell lymphomas in
susceptible birds.  Vaccination is the primary control measure,
which is effective in reducing incidence of tumor formation.
However, since MD vaccines are not sterilizing, they do not
prevent infection or horizontal spread of the virus.  As a
consequence, MDV field strains that overcome vaccinal protection
have arisen repeatedly over time.  Therefore, there is a need for
sustainable alternative controls measures, such as improving
genetic resistance.

Many studies have reported strong associations between MHC
alleles and resistance or susceptibility to MD.  For example,
chickens with MHC allele B\textsuperscript{21} are highly
resistant in contrast to chickens with B\textsuperscript{19}
allele which are highly susceptible.  ADOL lines 6 and 7, both
share (B\textsuperscript{2}) allele, yet exhibit different
phenotypic responses; e.g., challenge with JM/102W strain result
in 0 and 100\% MD incidence for lines 6 and 7, respectively.
Thus, the major unanswered questions are what genetic factors,
especially those that are non-MHC, contribute to susceptibility
and resistance of the disease and what is the main contributing
mechanism.

In the past decades, significant efforts have been to study
variations in a global gene expression between resistant and
susceptible birds using microarray and RNA-Seq methods in order
to identify non-MHC genes that contribute to resistance to
MD~\cite{sarson2008transcriptional,morgan2001induction,vallejo1998genetic,yonash1999high,bumstead1998genomic}.
However, none of the studies have investigated differential
expression of alternative isoforms, which are known to play a
significant role in many biological events including immune
responses.  In addition, studies have shown that isoform
expression levels can provide better signatures for some
diseases~\cite{zhang2013isoform}.  Changes of isoform expression
levels are governed partly by two types of {\em cis-}regulatory
elements: Exon Splicing Enhancer (ESE) and Exon Splicing Silencer
(ESS), which are located within an exon sequence.  A number of
sequence motifs of ESE and ESS have been identified in human and
some other organisms and could be predicted {\em in silico}.
Mutations that disrupt or create those motifs could alter
splicing patterns leading to aberrant alternative splicing.  A
number of disease-associated single-nucleotide polymorphsims in
coding regions (SNPs) that affect ESEs and ESSs have been well
characterized~\cite{blencowe2000exonic, wang2007splicing}.
%In fact, 15\% of mutations that cause genetic diseases affect
%pre-mRNA splicing via this mechanism.
Therefore, variations in isoform expression could lead to
identification of SNPs that underlie genetic resistance to MD.
In this article, we reported differential-expressed genes and
isoforms that may contribute to resistance to MD as well as SNPs
that can potentially affect isoform expression levels.

% Results and Discussion can be combined.
\section*{Results and Discussion}

\subsection*{Differential expression results from our method are comparable to
previous studies}

To extensively study gene and isoform expression, we incorporated
ENSEMBL gene models release 73 with {\em de novo} and a
reference-guided transcriptome assembly to build custom gene
models.  The models, therefore, include both ENSEMBL annotated
transcripts and putative genes and isoforms.  The advantage of
using custom gene models is it allows an investigation of
unannotated genes and isoforms, which is necessary for in-depth
study of an immune system.

Some DE genes have been reported from previous microarray studies
were also found to be differentially expressed in this study.
For example, B6.1 (Bu-1) is known to be down-regulated
approximately 2.27 fold in susceptible chickens with the MHC
allele B\textsuperscript{19} at 4 days post infection
(d.p.i)~\cite{sarson2008transcriptional}.  It was also found to
be down-regulated about 3-fold in the susceptible line in our
study.  Similarly, {\em GMZA} reported to be upregulated across
genetically different chickens (B\textsuperscript{19},
B\textsuperscript{21} alleles), was also found to be highly
upregulated.  In contrast, some genes that have been reported to
be highly expressed in resistant chickens were downregulated in
both lines. Those genes are {\em AMIGO2, MMP13} and {\em CLEC3B},
which were found to be downregulated more than
2-fold~\cite{sarson2008transcriptional}.
% Several genes involved in innate immune response were
% differentially expressed in resistant chickens: \textit{DDT,
% NMU, VIPR2} and \textit{HSP5}. Of all those genes, only HSP5
% was upregulated.
Other immune genes reported to be highly expressed in susceptible
chickens including {\em AVD, ART1, NOS2, CXCL13L2, MX1} and {\em
SOCS1}~\cite{smith2011systems} were also found to be highly
upregulated in both lines.  However, our results show the similar
expression patterns for {\em IL6} and {\em IL18}, which were only
upregulated in the susceptible line in early stage of infection
($3-5$ d.p.i).

In contrast, {\em IL15} has been reported non
differential-expressed between control and infected chickens in
both lines~\cite{kaiser2003differential}; however, it was only
upregulated in the susceptible line.  Expression of {\em IL15} is
induced by {\em TLR9}, which binds to non-methylated CpG residues
present in genomes of many DNA viruses, including herpes simplex
virus.  This cytokine auto-regulates the expression of {\em
CD40}, which is a transmembrane receptor required for activation
of macrophages by CD4 T cells.  Consequently, {\em CD40} was only
upregulated in the susceptible line (data not shown).

\subsection*{Differential gene expression indicates active immune
responses to ongoing lytic infection in the susceptible line}

Many genes were found to be differentially expressed (DE) between
control and infected chickens in both lines.  While the number of
unique downregulated genes in both lines was approximately equal,
the number of unique upregulated genes in the susceptible line
was much greater compared to the resistant line
(Figure~\ref{degenes_venn}).  Interestingly, some genes that were
differentially expressed in both lines were regulated in the
opposite direction (Table~\ref{tab:opposite}).  Among genes
downregulated in the resistant line but upregulated in the
susceptible line were {\em LL} (lung lectin) and {\em SFTPA1},
which encode a calcium-dependent C-type lectin and a lung
surfactant protein respectively.  Both molecules are important in
innate immunity~\cite{hogenkamp2008chicken,kingma2006defense}.
{\em LIMS1} is involved in cell differentiation and proliferation
and {\em PPARG} is a suppressor of {\em NF$\kappa$B}-mediated
proinflamatory response.  On the other hand, nearly all genes
upregulated in the resistant line but downregulated in the
susceptible line are involved in cell survival such as mRNAs
splicing, cell growth, and protein synthesis, except CD7 whose
function is involved in T cell-B cell interaction.  This
difference suggests that even at this stage of infection in the
resistant line, the lytic phase could be subsided, therefore,
only genes important for survival of cell are upregulated in the
resistant line.  In comparison, the lytic phase in the
susceptible line may still continue and as a result, genes
involved in immune responses are still upregulated.  In addition,
type I interferon ({\em IFN-$\gamma$} and {\em IFN-$\beta$}) as
well as {\em INF-$\alpha$3} were found to be highly upregulated
in infected chickens in both lines  (Table~\ref{tab:cytokines}).
However, expressions of genes encoding their corresponding
receptors were not different in the resistant line, but
upregulated in the susceptible line.  This could also reflect the
ongoing immune responses in the susceptible line.

\subsection*{Functional analysis of differential-expressed genes
indicates inactive adaptive immune responses in the resistant
line}

To determine pathways that were perturbed during the infection,
data were analyzed by GOSeq, which accounted for gene lengths
bias unique for RNA-Seq data~\cite{young2010method}.
Significantly perturbed pathways (FDR $< 0.1$) from both lines
that involved in immune response include TLR signaling pathway,
cytokine-cytokine receptor interaction, intestinal immune network
for IgA production, and cell adhesion molecules
(CAMs)(figure~\ref{line67_kegg}).  Some other pathways important
in response to viral infection and only significantly enriched in
the susceptible line include phagosome, apoptosis, RIG-I-like
receptor signaling pathway, NOD-like receptor signaling pathway,
and lysosome.  Figure~\ref{kegg_phagosome} is a pictorial example
of genes in the phagosome pathway.  Although MHC I ({\em BF1})
was differentially expressed in both lines, other genes involved
in expressing newly synthesized MHC I were only upregulated in
the susceptible line suggesting that new MHC I molecules were
actively produced.  Furthermore, gene ontology analysis of
biological processes (GO:BP) shows that categories involved in
both adaptive and innate immune responses were enriched in the
susceptible line (supplementary material).  On the other hand,
only categories involved in innate immune responses were enriched
in the resistant line.  In addition, enrichment of apoptosis
pathway in the susceptible line indicates that the programmed
cell death could be induced by CTL response to eliminate the
viruses.

At this stage of infection, our results suggest that lytic
infection of MDV stimulates both innate and adaptive immune
responses, which leads to activation of T cells in the
susceptible line.  Only activated T cells are believed to be
infected by MDV, therefore, the lytic phase could facilitate the
spread of the viruses by enhancing expansion of activated T
cells.  Due to cell-associate nature of MDV, the viruses transfer
to T cells via cell-to-cell contact between B cells and T cells
during antigen presentation or B cells activation by T helper
cells.  Therefore, it is beneficial for the host to restrain such
contact.  However, it is not clear how chickens in the resistant
line control the lytic infection of MDV.  Two mechanisms have
been speculated to contribute to MD resistance.  First innate
immune responses could be highly effective and could activate
strong adaptive immune responses that rapidly control viral
replication and force the viruses to undergo latent phase.
Second, innate immune resposnes itself could be highly effective
in limiting viral replication~\cite{smith2011systems}.

\subsection*{Genes with differential exon usage (DEU) in response
to MDV infection can be divided into four groups based on their
patterns of expression}

The immune system is isoform-rich and many genes express
different isoforms with distinctive functions in response to
stimuli such as stress, chemicals and infection.  Changes in
expression of splice forms of immune related genes have been
reported to be associated with increase susceptibility and poor
prognosis of diseases~\cite{lynch2004consequences}.  Studying
differential isoform expression could therefore shed lights into
inherent differences between lines that confer resistance or
susceptibility to MD.

In the past decades, microarray technology has been used to study
gene and isoform expressions in many studies, but its sensitivity
for detection of structurally similar isoforms is low, and known
or predicted annotations are required to design
probes~\cite{kane2000assessment}.  Although RNA-Seq method can
provide a reliable estimate of an exon expression compared to
microarray~\cite{pan2008deep} and is not constrained to those
limitations, studying isoform expressions using RNA-Seq is still
not straightforward because of the short read lengths.  Reads
from current Illumina technology are generally not long enough to
span across all exons in an isoform.  In most cases, only exons
in close proximity are covered by the same read, which makes it
difficult to accurately predict a full structure of the isoform.
In addition, some genes are fused due to overlapping untranslated
regions (UTRs), which can also result in erroneous predicted
isoform structures.

Due to those issues, it is not feasible to accurately estimate
expression of isoforms, especially when gene annotation is
constructed from {\em de novo}
assembly~\cite{trapnell2013differential}.  To avoid these issues,
we chose to study exon expression instead of isoform expression.
Using MISO with exon-centric method, only reads spanning across a
few exons are used and only exons involved in a splicing event
are examined.  The expression of exon inclusion is calculated as
Percent Spliced In (Psi or $\Psi$), which can be used to infer
the portion of transcripts that include the exon in each
sample~\cite{Katz:2010iv}.  In this study, we investigated the
three most common alternative splicing events in vertebrates,
which are skipped exons (SE), an alternative $3\prime$ (A3SS) and
$5\prime$ (A5SS) splice site.  Lists of DEU genes from the
resistant line that show difference in $\Psi$ greater than 0.20
when compared to the susceptible line in infected chickens are
shown in
Tables~\ref{tab:line67i_diff_line67u_one},~\ref{tab:line67i_diff_line67u_two}
and~\ref{tab:line67i_diff_line67u_three}.  Genes can be
categorized roughly into four groups based on the pattern of
$\Psi$ across control and infected birds in both lines.

Group I (Table~\ref{tab:line67i_diff_line67u_one}) includes genes
with $\Psi$s that were up- or down-regulated in infected chickens
in the resistant line only.  This group includes {\em BCL11B}
(B-cell CLL/lymphoma 11B zinc finger proteins), a B-cell lymphoma
associated C2H2-type zinc finger protein encoding gene, which
functions as a tumour-suppressor in T-cell lymphoma in human.
According to homologous alignments on UCSC genome browser, a
splice form with the skipped exon is similar to mouse {\em BCL11B
isoform b}.  The skipped exon was expressed 30\% in infected the
resistant line chickens; whereas it was rarely expressed (4-7\%)
in the control the resistant line and both groups in the
susceptible line.  The skipped exon was not found to encode any
known protein domain, however, it is in the middle of two
adjacent C2H2-type finger protein domains.  Another gene related
to B and T cell lymphomas is {\em SIK2} (salt-inducible kinase
2).  This gene has been reported to have a negative effect on T
cell lymphomas by limiting the transcription of HTLV-1
virus~\cite{tang2013lkb1}.  Some other genes are important in
pre-mRNA splicing including {\em TRA2A}, {\em SRSF6} and {\em
GEMIN6}.  {\em TRA2A} (transformer-2 protein homolog alpha)
encodes RNA recognition motif (RRM), the skipped exon is not
found to encode any known protein domain.  {\em SRSF6} (SR
splicing factor 6) encodes a nuclear protein that belongs to the
splicing factor protein family.  {\em GEMIN6} plays a role in the
assembly of spliceosomal snRNP in cytoplasm.  These genes could
play a significant role in regulating inclusion of alternative
exons.  A gene in this group that might be important for innate
immune responses is {\em RAC3} (Ras-related C3 botulinum toxin
substrate 3).  This gene encodes small GTPases, belonging to Ras
family, that regulates a wide variety of cellular events
including cell growth, cytoskeletal reorganization, and the
activation of protein kinases.  The role of small GTPases in
immune responses is discussed further below.

In group II (Table~\ref{tab:line67i_diff_line67u_two}), $\Psi$
values were relatively stable in control and infected chickens
within line, but not between lines.  Genes that could play an
important role in immune responses are {\em ITGB2} and {\em HCK}.
{\em HCK} transmits signal from cell surface receptors such as
{\em FCGR1A, FCGR2A, IL2, IL6, IL18}, and integrins ({\em ITGB1,
ITGB2}).  {\em ITGB2} (CD18) encodes subunit $\beta_{2}$ integrin
of {\em LFA-1} and {\em CR3} receptors.  {\em LFA-1} plays an
important role in adhesion of lymphocytes with other cells.  {\em
CR3} binds to a vast array of ligands and molecules including
complement C3bi, microbial proteins, ICAM-1 and -2, ECM proteins,
and coagulation proteins~\cite{}.  It plays a significant role in
neutrophils and monocytes activation including phagocytosis,
adhesion and migration~\cite{}.  The role of {\em ITGB2} in
immune responses is discussed further in the next section.  {\em
DYNLT1,DYNLL2, SEPT11, PFN2}, and {\em ZDHHC7} are also involved
in cell rearrangement and cytokinesis.  In particular, {\em
DYNLT1} and {\em DYNLL2} are dynein proteins that have been
demonstrated to regulate T cell activation by driving T cell
receptor microclusters (TCR-MCs) toward the center of immune
synapse~\cite{hashimoto2011dynein}.

Group III (Table~\ref{tab:line67i_diff_line67u_three}) includes
genes that exhibit differential isoform expression only in
infected the susceptible line.  A number of genes in this group
encode proteins that are parts of spliceosome: {\em SRSF3}, {\em
HNRNPDL}, {\em SFSWAP}, {\em THOC1}, {\em RNPC3} and {\em SRSF5}.
Two genes involved in cell-cell contact regulated by integrins
are also in this group.  {\em PPP1R12A} is a myosin phosphatase
that regulates the interaction of actin and myosin downstream of
GTPase Rho proteins.  {\em PODXL} encodes PODX-like protein that
functions in integrin-dependent manner as both pro-adhesive and
anti-adhesive molecules.  This protein is involved in
cell-to-cell contact, cell trafficking, and cancer
progression~\cite{nielsen2009role,somasiri2004overexpression}.

The last group (Group IV,
Table~\ref{tab:line67i_diff_line67u_three}) only has one gene,
{\em GOSR1}.  The $\Psi$ value of this gene was less than 0.20
cutoff in control and infected chickens in the resistant line,
but it is significantly different between infected chickens in
the resistant and susceptible lines.  Also, there is a significant
difference between control and infected chickens in the
susceptible line.  This gene encodes a trafficking membrane
protein important for transporting proteins from {\em cis-} to
{\em trans-}golgi network.

\subsection*{Roles of {\em LFA-1} and actin cytoskeleton in T
cells activation}

By grouping genes based on patterns of $\Psi$s, we found that
many genes in group II: {\em ITGB2, PFN2, DYNLL2, DYNLT1,
SEPT11}, and {\em ZDHHC7}, are involved in cytokinesis or cell
synapse, which are important for T cell activation.  As described
above, {\em ITGB2} encodes $\beta$-subunit of integrins including
LFA-1, which exclusively expressed in lymphocytes and plays a
major role in lymphoproliferation, antigen presentation, T cells
activation, and cytotoxicity.  Integrins are a special kind of
receptors that transmit signals bidirectionally across the cell
membrane.  They are heterodimeric composed of an $\alpha$ (large)
and a $\beta$ (small) subunits~\cite{wang2010immunopathologies}.
$\beta_{2}$ (CD18) subunit encoded by {\em ITGB2} is expressed on
lymphocytes and antigen presenting cells (APCs) as a component of
{\em LFA-1} and {\em CR3} receptors.  LFA-1 binds to its ligand
ICAM-1 to help form a synapse that bring APC and T cell together
to initiate antigen presentation leading to T cell
activation~\cite{dustin2000immunological}.

Absence of LFA-1 leads to impaired functions of lymphocytes in
proliferation and tumor
rejection~\cite{scharffetter1998spontaneous,schmits1996lfa}.
Mutations in {\em ITGB2} gene has been associated with type 1
leukocyte adhesion deficiency (LAD-1), an autosomal-recessive
inherited disease found in few families.  The disease is
characterized by impair of lymphocytes in adherent-dependent
functions, lack of accumulation to the site of infection and
recurrent bacterial and fungal
infection~\cite{springer1987lymphocyte}.  In addition, the
response of lymphocytes to mitogens is decreased in patients with
LAD~\cite{springer1987lymphocyte}.  The decrease in
responsiveness to mitogens have been shown to correlate with
resistance to MD by Lee and Bacon~\cite{lee1983ontogeny}, who
illustrated that resistant birds (the resistant line and N) were
less responsive to phytohemagglutinin (PHA) than susceptible
birds (line 7 and P).

Actin cytoskeleton is very important in T cell activation because
it enhances activity of LFA-1 by increasing its avidity and
recruiting signaling molecules necessary for downstream
signaling~\cite{dustin2000immunological, van2000avidity}.
Cytoskeleton proteins binding to cytoplasmic domain of LFA-1 are
thought to play an important role in driving LFA-1 to aggregation
on the cell surface, resulting in increased avidity.  Aggregation
of LFA-1 has been demonstrated to be essential for lymphocytes to
bind to the ligand~\cite{van1994extracellular}.  Interestingly,
{\em RAC3, PFN2,} and {\em PPP1R12A}, which are involved in actin
cytoskeleton pathway (Figure~\ref{kegg_actin}), also expressed
different ratios of alternative splice forms between lines .
Some of these genes are also co-present in three other pathways
that involved in immune responses (Table~\ref{tab:integrin}).  It
could be speculated that pre-mRNA splicing of these genes is
co-regulated by splicing regulators or some genetic factors.

\subsection*{Prediction of functional domains of splice forms of
genes in the actin cytoskeleton pathway}

To predict the function of the alternative splice forms of genes
in actin cytoskeleton pathway, transcript sequences were
translated to protein sequences by
ESTscan~\cite{iseli1999estscan}.  Protein sequences were then
searched for annotated protein domains using the standalone
version of InterPro Scan~\cite{quevillon2005interproscan}.
Besides {\em ITGB2}, other genes have alternative exons located
in coding regions and could potentially affect functional protein
domains in some ways.  The exon with alternative 3$\prime$ splice
site of {\em RAC3} encode part of a protein domain identified as
small GTPase of Ras subfamily (ProSiteProfiles:PS5142 and
SMART:SM00173).  Rac3 is highly homologous to Rac1 and has been
reported to possess the ability of promoting membrane ruffling,
transformation, activation of c-Jun transcriptional activity and
a co-activator of NF$\kappa$B~\cite{werbajh2000rac}.  Activated
Rac also regulates production of superoxide in neutrophils and
macrophages.

The alternative exon of {\em PFN2} seems to disrupt the coding
sequence that encode profilin domain (Pfam:PF00235).  The
profilin domain is essential for almost all organisms and its
functions include regulating actin polymerization, controlling
complex network of molecular interaction and transmitting signals
from small-GTPase pathway.  It also binds to Rac effector
molecules and a number of other ligands~\cite{witke2004role}.
The skipped exon of {\em PPP1R12A} encodes part of a protein
domain annotated as protein phosphatase 1 (PIRSF:PIRSF038141).
The functional effects of alternative splicing on these
functional domains have to be further investigated by simulation
or an experiment.

Even though the exact mechanism is not known, incompetency of T
cells in response to stimuli appears to benefit resistant birds
because MDV could not induce T cells to proliferate and cause
them to undergo neoplastic transformation.  It has also been
suggested that the mechanism that controls both lymphocyte
proliferation induced by MDV and lymphocyte proliferation induced
by immune response is the
same~\cite{pazderka1975histocompatibility}.  Therefore, it may be
useful to consider a link between deficiency of lymphocytes in
the resistant line to the alternative splice form of {\em ITGB2}
that is only expressed in the resistant line.  Although the exon
included in the alternative splice form is non-coding, it could
serve important functions in translation or posttranscriptional
regulation.


% Based on our gene model, the skipped exon of \textit{ITGB2} is
% located in the 5$\prime$ UTR; whereas, the skipped exon of
% \textit{HCK}, encodes protein tyrosine kinase (Pfam:PF07714).

\subsection*{Prediction of {\em cis}-regulatory elements in
alternative splicing exons of genes in group II}

Although alternative splicing of genes in all groups could be
regulated by {\em cis-} and {\em trans-}splicing factors, genes
in group II are more likely to be regulated by {\em
cis-}regulatory factors.  The ratios of isoform expression in
this group were relatively stable within line, but were
significantly different between lines.  Investigation of
nucleotide differences within exons of both lines could reveal a
possible role of SNPs in regulating alternative splicing in this
group.  We obtained a sequence of an alternative exon from the
resistant line and used Human Splicing Finder (HSF) to determine
whether SNPs from the susceptible line could alter predicted ESEs
or ESSs.  Results from some genes involved in cytokinesis are
discussed in this section.  Exonic SNPs from the resistant and
susceptible lines are listed in Table~\ref{tab:deu_snps}.

For {\em ITGB2}, a SNP (T) at position 26 of the cDNA from the
resistant line, which corresponds to position 7,183,696 on
chromosome 7 is located in a predicted binding site for SC35,
which is an exon enhancer.  Although the exon is not expressed in
the susceptible line, we found that there is no polymorphisms
between lines according to SNPs data from the genome
resequencing.  Therefore, this SNP may not be accounted for
exclusion of the exon in the susceptible line
(Figure~\ref{itgb2}).  Exon sequences of {\em PFN2} from the
resistant line and 7 differ at position 23,221,934 on chromosome
9.  A small insertion of two AA nucleotides is predicted to
create a new binding site for Tra2-$\beta$ splicing regulator,
which serves as a stabilizer of an enhancer
complex~\cite{lopez1998alternative}.  From exon expression data,
$\Psi$ of an exon with alternative splicing increases from
0.20-0.30 to about 0.50 in the susceptible line.  Tra2 could
possibly increase inclusion of the exon with alternative
3$\prime$ splice site via ESE-dependent 3$\prime$ splice site
activation (Figure~\ref{pfn2}).  Even though the linear distance
between Tra2-$\beta$ binding site and the alternative 3$\prime$
splice site is greater than 1kb, Tra2 could possibly get close to
the 3$\prime$ splice site in the secondary structure of the mRNA.

Replacement of A with G nucleotide in the skipped exon of {\em
DYNLL2} from line 6 is predicted to slightly alter the binding
site of several ESEs as well as to create a new binding site for
9G8.  This exon is upregulated in the susceptible line compared
to the resistant line, therefore, the present of the new binding
site for 9G8 exon enhancer helps support the expression results.
In addition, A nucleotide is this position matches the reference
nucleotide, therefore, we could expect this exon to be expressed
in other datasets.  According to EST tags on the UCSC genome
browser, the exon has been found and sequenced from chicken eyes
(15d post-hatched, EST sequence:DR424100).  For {\em DYNLT1}, the
skipped exon was absent in the susceptible line, therefore, we
could not determine the SNPs from RNA-Seq data.  However, data
from DNA resequencing showed that there was a polymorphism at
position 51,357,865 on chromosome 3.  Replacing G with C from the
susceptible line is predicted to slightly alter predicted binding
sites for 9G8 as well as SRp55 exon enhancer.  However, whether
the alteration down regulates the exon in line 7 is unclear.

There are too many SNPs in the exon of {\em SEPT11} and {\em
ZDHHC7}, making it unfeasible to predict which SNP regulates the
exon expression.  Therefore, we do not discuss these two genes in
this section.  However, results from HSF analysis of these two
genes and other genes in this group are provided in the
supplementary materials.  Experimental validation of exonic SNPs
provided by this study could shed some lights on underlying
polymorphisms that contribute to

\section*{Conclusion}

Custom gene models built from combination of gene models from
{\em de novo} assembly, reference-based assembly, and ENSEMBL had
allowed us to identify genes and isoforms that might play an
important role in resistance to MD.  Results from gene expression
analysis indicated that adaptive immune responses were active
during lytic infection in the susceptible line, but not in the
resistant line.  Because only activated T cells are infected by
MDV, we speculated that elicitation of adaptive immune responses
could help spread the viruses by recruiting and activating more T
cells and.  In contrast, the delay of adaptive immune responses
could benefit the host by limiting infection of activated T
cells.

To elucidate the molecular mechanism of MD resistance, we
investigated differential isoforms expression between lines  and
identified a number of genes that could be responsible for
difference in immune responses. Notably, several genes are
involved in actin cytoskeleton and cytokinesis, which are
important for the functions of lymphocytes and immune cells but
have not been of great interest in the filed of MD research.
Even though we mainly discussed the possible role of {\em ITGB2}
in MD resistance, other genes cannot be precluded and should be a
candidate for further investigation and experimental studies.
Moreover, a full mechanism of MD resistance is highly complex and
more data from different stages of infection as well as a greater
sequencing depth will be required to identify all genes and
isoforms involved.  To enable the study of unannotated gene and
isoform expression, our approach of constructing gene models from
RNA-Seq should be iteratively used to extend the ENSEMBL gene
models to construct more complete gene models.

% You may title this section "Methods" or "Models". 
% "Models" is not a valid title for PLoS ONE authors. However, PLoS ONE
% authors may use "Analysis"
\section*{Materials and methods}

\subsection*{Sequences and quality trimming}

mRNAs were extracted from spleens of control and infected
chickens lines 6 and 7 (4 d.p.i).  Sequence libraries were
prepared by standard Illumina unstranded single- and paired-end
protocols.  Library size of the paired-end datasets is
approximately 175 bp.  Read lengths are 75 bp in both single- and
paired-end libraries.  Reads were quality trimmed by Condetri
2.1~\cite{smeds2011condetri} with quality score cutoff of 30.
The first 10 bases were removed due to non-uniform distribution
of nucleotides.

\subsection*{Gene models construction}

Due to lack of complete gene models for chickens, we employed two
methods to construct gene models from RNA-Seq reads.  First,
short reads were assembled using
Velvet/1.2.03~\cite{Zerbino:2008vu} and
Oases/0.2.06~\cite{Schulz:2012je} to obtain long transcripts.
Assembly was done with hash lengths range from 21 to 31 for both
local and global assembly (described in Gimme paper).  Poly-A
tails were trimmed and low complexity transcripts were removed by
Seqclean~\cite{seqclean}.  All transcripts were then aligned to
chicken reference genome (galGal4, with unplaced and random
chromosomes removed) with BLAT~\cite{Kent:2002tv}.  Second, reads
were aligned to reference genome using
Tophat/2.0.9~\cite{Trapnell:2009dp} and ENSEMBLE gene models
release 73 was used to guide reference-based assembly by
Cufflinks2~\cite{Trapnell:2010kd}.  Alignments from BLAT and
models from Cufflinks were then combined to construct gene models
by Gimme (manuscript in preparation).

\subsection*{Differential gene expression analysis and gene
ontology}

To identify DE genes, reads were mapped to transcripts by RSEM
v.1.2.7~\cite{li2011rsem}, which is also used to estimate genes
expression and identify DE genes.  Data from single- and
paired-end datasets from the same line were treated as biological
replicates.  To identify enriched pathways and ontology terms, a
list of DE genes was analysed by GOSeq v.1.10.0 based on chicken
KEGG annotations.  P-values were corrected by Benjamini-Hochberg
multiple testing correction.  Genes, patways and GO terms with
corrected P-value $<0.1$ were considered significant.
Pathview~\cite{luo2013pathview} was used to create a KEGG pathway
diagram with colors representing relative level of gene
expressions.

\subsection*{Differential exon usage analysis}

Gene models were converted to alternative splicing models using a
Python script.  In order to increase sensitivity, read counts
from single- and paired-end samples were combined and treated as
single-end reads for splicing event analysis with
MISO/0.4.9~\cite{Katz:2010iv}.  Splicing events with Bayes factor
$>10$ and $\Delta\Psi>0.20$ were considered significant.  Read
coverages and $\Psi$ distributions were plotted using Sashimi
plot~\cite{Katz:2013vx}.

\subsection*{Variant calling and {\em in silico} splicing
analysis}

Variants were called using mpileup command from
SAMTools/0.1.18~\cite{li2009sequence} and
BCFTools~\cite{bcftools}.  Only variants with quality score
$\ge20$ were used for mutation analyses.  Exon enhancers and
suppressors were predicted using Human Splicing Finder web
portal~\cite{desmet2009human}.  Human default parameter settings
were used in all analyses. 

\subsection*{Protein domains search}

Transcripts were translated to protein sequences using ESTScan
3.0.3~\cite{iseli1999estscan} with chicken HMM matrices built
from chicken cDNAs and refSeq sequences. Protein sequences were
searched against InterPro database using
InterProScan/5.44.0~\cite{quevillon2005interproscan}.

% Do NOT remove this, even if you are not including acknowledgments
\section*{Acknowledgments}

% \section*{References}
% The bibtex filename
\bibliography{ref}{}

\section*{Figure Legends}

\begin{figure}[!ht]
    \begin{center}
        \includegraphics[width=6in]{degenes_venn.pdf}
    \end{center}
    \caption{
        {\bf Differential-expressed genes in response to MDV infection}
    }
    \label{degenes_venn}
\end{figure}

\begin{figure}[!ht]
    \begin{center}
        \includegraphics[width=7in]{line67_KEGG_cleveland.pdf}
    \end{center}
    \caption{
        {\bf Enriched KEGG pathways.}
    }
    \label{line67_kegg}
\end{figure}

\begin{figure}[!ht]
    \begin{center}
        \includegraphics[width=6in]{gga04145_degenes_multi.png}
    \end{center}
    \caption{
        {\bf Phagosome pathway.}
    }
    \label{kegg_phagosome}
\end{figure}

\begin{figure}[!ht]
    \begin{center}
        \includegraphics[width=6in]{hsa04810_deu_genes.png}
    \end{center}
    \caption{
        {\bf Human regulation of actin cytosekeleton pathway.}
    }
    \label{kegg_actin}
\end{figure}


\begin{figure}[!ht]
    \begin{center}
        \includegraphics[width=6in]{itgb2_miso.pdf}
    \end{center}
    \caption{
        {\bf ITGB2 exon expression.}
        A single nuclotide substition (T$>$C) at position
        7,183,696 is predicted to disrupt a putative ESE motif
        which binds SC35 on the skipped exon of IRGB2 gene shown
        in this figure.  The prediction correlate with exclusion
        of the exon in the susceptible line ($\Psi \leq 0.02$).
    A new binding site for a silencer is also predicted and could
also promote the exclusion of the exon.  }
    \label{itgb2}
\end{figure}

\begin{figure}[!ht]
    \begin{center}
        \includegraphics[width=6in]{pfn2_miso.pdf}
    \end{center}
    \caption{
        {\bf PFN2 exon expression.}
        A small insertion of AA nucleotides at position
        23,221,934 is predicted to create a putative ESE motif
        which binds to Tra2 protein.  Tra2 could promote
        inclusion of exon with alternative 3$\prime$ splice site
        via ESE-dependent 3$\prime$ splice site activation
        resulting in higher expression of the exon in the
        susceptible line.  Even though the predicted ESE motif is
        located at a substantial distance from the alternative
        3$\prime$ splice site, Tra2 could still regulate the
    splicing if a secondary structure of the mRNA moves the ESE
motif closer to the alternative 3$\prime$ splice site.  }
    \label{pfn2}
\end{figure}

\section*{Tables}

\begin{table}[!ht]
\caption{
\bf{Genes regulated in opposite directions}}
\begin{tabular}{cccc}
        \hline
        Gene & Description & $log_{2}$FC & \\
         & & Resistant & Susceptible\\
        \hline
        LL & Lung lectin & -3.36 & 8.70 \\
        GIF & Gastric intrinsic factor & -2.15 & 3.11 \\
        SFTPA1 & Surfactant protein A1 & -4.84 & 3.73 \\
        SCAF8 & SR-related CTD-associated factor 8 & -8.70 & 8.24 \\
        LIMS1 & LIM and senescent cell antigen-like domains 1 & -2.26 & 1.33 \\
        PPARG & Peroxisome proliferator-activated receptor gamma & -6.99 & 2.06 \\
        C14ORF1 & Chromosome 14 open reading frame 1 & -3.11 & 2.71 \\
        Unknown & Unknown & -3.08 & 1.62 \\
        \hline
        ATP8A2 & ATPase, aminophospholipid transporter, class I, type 8A, member 2 & 7.66 & -7.80 \\
        S1PR1 & Sphingosine-1-phosphate receptor 1 & 1.31 & -6.95 \\
        MED9 & Mediator complex subunit 9 & 8.22 & -3.15 \\
        DNAJA2 & DnaJ (Hsp40) homolog, subfamily A, member 1 & 2.34 & -6.02 \\
        RAD17 & RAD17 homolog (S. pombe) & 1.45 & -1.80 \\
        PSMG3 & Proteosome assembly chaperone 3 & 1.52 & -1.26 \\
        PNISR & PNN-interacting serine/argining-rich protein & 5.58 & -1.81 \\
        THOC7 & THO complex 7 homolog (Drosophila) & 7.00 & -6.90 \\
        YTHDC2 & YTH domain containing 2 & 1.00 & -1.74 \\
        RPL39 & Ribosomal protein L39 & 3.31 & -1.59 \\
        CD7 & CD7 molecule & 2.88 & -1.38 \\
        NDUFB3 & NADH dehydrogenase (uqiquinone) 1 beta subcomplex 3 & 1.03 & -1.34 \\
        LOC100858785 & Unknown & 1.26 & -1.75 \\
        Unknown & Unknown & 6.30 & -5.22 \\
        \hline
    \end{tabular}
    \begin{flushleft}
        (-) down-regulated, (+) up-regulated
    \end{flushleft}
    \label{tab:opposite}
\end{table}

\begin{table}[!ht]
\caption{
\bf{Cytokine-related gene expression in response to MDV infection}}
    \begin{tabular}{cccccc}
        \hline
        & & $log_{2}$FC & \\
        Symbol & Description & Resistant & Susceptible \\
        \hline
        IL2RG & Interleukin 2 receptor, $\gamma$ & -- & 0.55 \\
        IL6 & Interleukin 6 (interferon, $\beta$ 2) & -- & 5.15 \\
        IL6ST & Interleukin 6 signal transducer (gp130, oncostatin M receptor) & -- & 1.36 \\
        IL8L1 & Interleukin 8-like 1 & 1.90 & -- \\
        IL15 & Interleukin 15 & -- & 1.06 \\
        IL18 & Interleukin 18 (interferon-$\gamma$ inducing factor) & 1.92 & 4.06 \\
        IL18R1 & Interleukin 18 receptor 1 & 1.94 & 1.64 \\
        IFNG & Interferon-$\gamma$ & 5.14 & 4.90 \\
        IFNB & Interferon-$\beta$ & 4.83 & 5.64 \\
        IFNA3 & Interferon-$\alpha$ 3 & 4.09 & 5.48 \\
        IFNGR1 & Interferon-$\gamma$ receptor 1 & -- & 2.05 \\
        IFNGR2 & Interferon-$\gamma$ receptor 2& -- & 0.50 \\
        IFNAR1 & Interferon-$\alpha$,$\beta$ receptor 1 & -- & 1.46 \\
        IFNAR2 & Interferon-$\alpha$,$\beta$ receptor 2 & -- & 0.58 \\
        \hline
    \end{tabular}
    \begin{flushleft}
    \end{flushleft}
    \label{tab:cytokines}
\end{table}

% \begin{table}[!ht]
%     \caption{
%     \bf{Gene models summary}}
%     \begin{tabular}{ccc}
%         \hline
%         Method & Gene & Isoform \\
%         \hline
%         Assembly & 25,290 & 54,044 \\
%         Cufflinks & 21,345 & 36,218 \\
%         Combined & 24,980 & 46,613 \\
%         \hline
%     \end{tabular}
%     \begin{flushleft}
%         Number of genes and isoforms from gene models
%         built from {\em de novo} assembly and Cufflinks.
%         Combination of both methods decreased the number of genes
%         and isoforms by merging fragmented transcripts to
%         form more complete gene models.
%     \end{flushleft}
%     \label{tab:gene_models}
% \end{table}


\begin{table}[!ht]
\caption{
\bf{DEU between the resistant line and the susceptible line in infected birds group I}}
\begin{tabular}{cccccccc}
\hline
& & & & Resistant ($\Psi$) & & Susceptible ($\Psi$) & \\
Type & Event ID & Ensembl & Symbol  & Un & Inf & Un & Inf \\
\hline
SE & chr2:13729.ev2 & ENSGALG00000010973 & \textbf{TRA2A} & 0.45 & \textbf{0.75} & 0.58 & 0.50 \\
SE & chr5:22858.ev1 & ENSGALG00000011127 & BCL11B & 0.07 & \textbf{0.30} & 0.06 & 0.04 \\
SE & chr2:13065.ev1 & ENSGALG00000013137 & INO80C & 0.14 & \textbf{0.35} & 0.96 & 0.86 \\
SE & chr20:14995.ev1 & ENSG00000124193* & SRSF6 & 0.43 & \textbf{0.72} & 0.54 & 0.34 \\
A5SS & chr7:24049.ev1 & ENSGALG00000009824 & \textbf{C7H2ORF77} & 0.50 & \textbf{0.73} & 0.32 & 0.38 \\
A5SS & chr3:18098.ev1 & ENSGALG00000013821 & \textbf{GEMIN6} & 0.84 & \textbf{0.61} & 0.81 & 0.85 \\
A3SS & chr1:6450.ev1 & ENSGALG00000027665 & \textbf{SYNGR1} & 0.46 & \textbf{0.22} & 0.68 & 0.60 \\
A3SS & chr2:13270.ev1 & ENSGALG00000026498 & \textbf{Unknown} & 0.12 & \textbf{0.71} & 0.10 & 0.34 \\
A3SS & chr19:12399.ev1 & ENSGALG00000005685 & \textbf{KSR1} & 0.23 & \textbf{0.56} & 0.28 & 0.35 \\
A3SS & chr1:6437.ev1 & ENSGALG00000012050 & \textbf{TNRC6B} & 0.57 & \textbf{0.39} & 0.95 & 0.93 \\
A3SS & chr18:11800.ev1 & ENSGALG00000002859 & \textbf{RAC3} & 0.31 & \textbf{0.15} & 0.33 & 0.39 \\
\hline
\end{tabular}
\begin{flushleft}
    *human homologs, Un=uninfected, Inf=infected.
    Bold face indicates that there is a SNP between the resistant line and 7 within an alternative exon.
\end{flushleft}
\label{tab:line67i_diff_line67u_one}
\end{table}

\begin{table}[!ht]
\caption{
\bf{DEU between the resistant line and the susceptible line in infected birds group II}}
\begin{tabular}{cccccccc}
\hline
& & & & Resistant ($\Psi$) & & Susceptible ($\Psi$) & \\
Type & Event ID & Ensembl & Symbol  & Un & Inf & Un & Inf \\
\hline
SE & chr17:11370.ev2 & ENSGALG00000004971 & URM1 & \textbf{0.07} & \textbf{0.03} & 0.18 & 0.23 \\
SE & chr2:12495.ev3 & ENSGALG00000005582 & \textbf{KLHL18} & \textbf{0.44} & \textbf{0.27} & 0.56 & 0.52 \\
SE & chr7:24327.ev1 & ENSGALG00000007511 & \textbf{ITGB2} & \textbf{0.17} & \textbf{0.22} & 0.02 & 0.01 \\
SE & chr2:12984.ev1 & ENSGALG00000012809 & ECI2 & \textbf{0.15} & \textbf{0.29} & 0.58 & 0.49 \\
SE & chr8:25247.ev1 & ENSGALG00000028790 & DNASE2B & \textbf{0.38} & \textbf{0.50} & 0.29 & 0.26 \\
SE & chr1:4653.ev1 & ENSGALG00000011682 & CNOT4 & \textbf{0.57} & \textbf{0.63} & 0.40 & 0.40 \\
SE & chr20:14907.ev3 & ENSGALG00000006522 & \textbf{HCK} & \textbf{0.46} & \textbf{0.59} & 0.99 & 0.97 \\
SE & chr4:21539.ev1 & ENSGALG00000015709 & \textbf{TACC3} & \textbf{0.87} & \textbf{0.94} & 0.76 & 0.72 \\
SE & chr2:32201.ev1 & ENSGALG00000012258 & GOLGA4 & \textbf{0.33} & \textbf{0.34} & 0.51 & 0.61 \\
SE & chr3:19470.ev1 & ENSGALG00000019979 & DYNLT1 & \textbf{0.23} & \textbf{0.22} & 0.02 & 0.02 \\
SE & chr19:12391.ev1 & ENSGALG00000005522 & \textbf{DYNLL2} & \textbf{0.01} & \textbf{0.02} & 0.19 & 0.25 \\
A5SS & chr6:23812.ev1 & ENSG00000107651* & \textbf{SEC23IP} & \textbf{0.02} & \textbf{0.12} & 0.50 & 0.43 \\
A5SS & chr2:12775.ev1 & ENSGALG00000011488 & \textbf{CMTM7} & \textbf{0.55} & \textbf{0.67} & 0.37 & 0.41 \\
A3SS & chr8:25257.ev1 & ENSGALG00000008939 & FUBP1 & \textbf{0.58} & \textbf{0.74} & 0.41 & 0.46 \\
A3SS & chr9:25750.ev1 & ENSGALG00000010410 & \textbf{PFN2} & \textbf{0.71} & \textbf{0.78} & 0.53 & 0.50 \\
A3SS & chr4:21230.ev1 & ENSGALG00000011476 & \textbf{SEPT11} & \textbf{0.22} & \textbf{0.14} & 0.40 & 0.42 \\
A3SS & chr4:20185.ev1 & ENSGALG00000008507 & THOC2 & \textbf{0.48} & \textbf{0.47} & 0.31 & 0.22 \\
A3SS & chr11:8703.ev1 & ENSGALG00000020987 & \textbf{ZDHHC7} & \textbf{0.42} & \textbf{0.23} & 0.57 & 0.55 \\
A3SS & chr4:21136.ev1 & ENSGALG00000027908 & \textbf{LOC422528} & \textbf{0.72} & \textbf{0.61} & 0.87 & 0.91 \\
\hline
\end{tabular}
\begin{flushleft}
    *human homologs, Un=uninfected, Inf=infected.
    Bold face indicates that there is a SNP between the resistant line and 7 within an alternative exon.
\end{flushleft}
\label{tab:line67i_diff_line67u_two}
\end{table}

\begin{table}[!ht]
\caption{
\bf{DEU between the resistant line and the susceptible line in infected birds group III and IV}}
\begin{tabular}{cccp{2cm}cccc}
\hline
& & & & Resistant ($\Psi$) & & Susceptible ($\Psi$) & \\
Type & Event ID & Ensembl & Symbol  & Un & Inf & Un & Inf \\
\hline
SE & chr12:8987.ev1 & ENSGALG00000008320 & EDEM1 & 0.96 & 0.99 & 0.91 & \textbf{0.72} \\
SE & chr4:20411.ev1 & ENSGALG00000023199 & HNRPDL & 0.39 & 0.40 & 0.30 & \textbf{0.18} \\
SE & chrZ:26257.ev1 & ENSGALG00000001745 & PSTPIP2 & 0.06 & 0.04 & 0.13 & \textbf{0.26} \\
SE & chr1:6316.ev3 & ENSG00000058272* & PPP1R12A & 0.99 & 0.97 & 0.96 & \textbf{0.77} \\
SE & chr1:4478.ev2 & ENSGALG00000009029 & TSPAN12 & 0.09 & 0.15 & 0.20 & \textbf{0.46} \\
SE & chr4:20075.ev1 & ENSGALG00000006157 & DDX26B & 0.67 & 0.61 & 0.58 & \textbf{0.84} \\
SE & chr6:23515.ev1 & ENSGALG00000003861 & HERC4 & 0.31 & 0.37 & 0.45 & \textbf{0.06} \\
SE & chr26:16840.ev1 & ENSGALG00000000533 & SRSF3 & 0.36 & 0.38 & 0.30 & \textbf{0.16} \\
SE & chr11:8507.ev2 & ENSGALG00000000904 & \textbf{C11H16ORF57} & 0.91 & 0.98 & 0.84 & \textbf{0.78} \\
SE & chr1:4323.ev1 & ENSGALG00000006409 & PODXL & 0.21 & 0.34 & 0.26 & \textbf{0.13} \\
SE & chrZ:26582.ev2 & ENSGALG00000014642 & LOC374195 & 0.60 & 0.57 & 0.70 & \textbf{0.81} \\
SE & chr6:23833.ev1 & ENSG00000175029* & CTBP2 & 0.38 & 0.38 & 0.23 & \textbf{0.12} \\
A5SS & chr15:10720.ev1 & ENSGALG00000002487 & \textbf{SFSWAP} & 0.58 & 0.73 & 0.55 & \textbf{0.41} \\
A5SS & chr7:24350.ev1 & ENSGALG00000008038 & \textbf{SF3B1} & 0.41 & 0.57 & 0.55 & \textbf{0.31} \\
A3SS & chr2:13147.ev1 & ENSGALG00000014915 & \textbf{THOC1} & 0.33 & 0.48 & 0.33 & \textbf{0.23} \\
A3SS & chr23:15908.ev1 & ENSGALG00000000720 & LOC419563 & 0.12 & 0.06 & 0.17 & \textbf{0.30} \\
A3SS & chr8:25157.ev1 & ENSGALG00000005162 & RNPC3 & 0.45 & 0.64 & 0.58 & \textbf{0.33} \\
A3SS & chrZ:27197.ev1 & ENSGALG00000000189 & \textbf{YTHDC2} & 0.44 & 0.59 & 0.42 & \textbf{0.32} \\
A3SS & chr23:15983.ev1 & ENSG00000163875* & \textbf{MEAF6} & 0.28 & 0.57 & 0.40 & \textbf{0.29} \\
A3SS & chr5:21970.ev1 & ENSGALG00000009421 & \textbf{SRSF5} & 0.55 & 0.72 & 0.54 & \textbf{0.39} \\
\hline
SE & chr27:17351.ev1 & ENSGALG00000001107 & GOSR2 & 0.73 & 0.92 & 0.38 & 0.60 \\
\hline
\end{tabular}
\begin{flushleft}
    *human homologs, Un=uninfected, Inf=infected.
    Bold face indicates that there is a SNP between the resistant
    line and 7 within an alternative exon.
\end{flushleft}
\label{tab:line67i_diff_line67u_three}
\end{table}

\begin{table}[!ht]
\caption{
\bf{Pathways containing {\em RAC3, ITGB2, PFN2} and {\em PPP1R12A}}}
\begin{tabular}{ccc}
\hline
Pathway ID &  Description & Gene \\
\hline
hsa04810 & Regulation of actin cytoskeleton & {\em RAC3, ITGB2, PPP1R12A, PFN2} \\
hsa04015 & RAP1 signaling pathway & {\em RAC3, ITGB2, PFN2} \\
hsa04650 & Natural killer cells cytotoxicity & {\em RAC3, ITGB2} \\
hsa05416 & Viral myocarditis & {\em RAC3, ITGB2} \\
hsa04510 & Focal adhesion & {\em RAC3, PPP1R12A} \\
\hline
\end{tabular}
\begin{flushleft}
\end{flushleft}
\label{tab:integrin}
\end{table}

\begin{table}[!ht]
\caption{
\bf{SNPs in exons}}
\begin{tabular}{ccccccc}
\hline
Gene &  Chromosome & Position & Reference & Resistant & Susceptible & Strand \\
\hline
ITGB2 & 7 & 7183696 & C & {\em T} & . & - \\
PFN2 & 9 & 23221934 & -  - & . & {\em AA} & + \\
DYNLT1 & 3 & 51357865 & C & . & {\em G} & - \\
DYNLL2 & 19 & 8694148 & G & . & {\em A} & - \\
\hline
\end{tabular}
\begin{flushleft}
\end{flushleft}
\label{tab:deu_snps}
\end{table}

\begin{table}[!ht]
\caption{
\bf{Results from human splicing finder}}
\begin{tabular}{lp{1cm}p{3cm}llll}
\hline
Gene &  cDNA Position & Linked SR protein & Type & Reference Motif & Mutant Motif & Variation \\
\hline
% ITGB2 & 20 & SC35 & ESE$^{1}$ & TGCTCATG (78.19) & & Site broken \\
%  & 22 & SF2/ASF(IgM-BRCA1) & ESE$^{1}$ & CTCATGG (77.23) & CTCACGG (91.15) & +18.03\% \\
%  & 22 & SF2/ASF(IgM-BRCA1), SF2/ASF & ESE$^{1}$ & CTCATGG (77.23) & CTCACGG (89.23) & +15.53\% \\
%  & 22 & SF2/ASF, SF2/ASF(IgM-BRCA1) & ESE$^{1}$ & CTCATGG (74.55) & CTCACGG (91.15) & +22.27\% \\
%  & 22 & SF2/ASF, SF2/ASF & ESE$^{1}$ & CTCATGG (74.55) & CTCACGG (89.23) & +19.69\% \\
%  & 24 & SRp55 & ESE$^{1}$ & & CACGGA (79.30) & New site \\
%  & 26 & SF2/ASF(IgM-BRCA1) & ESE$^{1}$ & & CGGAGAT (80.00) & New site \\
%  & 26 & SF2/ASF & ESE$^{1}$ & & CGGAGAT (75.36) & New site \\
%  & 21 & & ESS$^{3}$ & & GCTCACGG (63.35) & New site (-5.59) \\
%  & 23 & & ESS$^{3}$ & TCATGGAG (61.41) & & Site broken (3.16) \\
%  & 23 & & ESS$^{3}$ & ATGGAGAT (64.83) & ACGGAGAT & -5.16\% \\
%  & 24 & & ESS$^{4}$ & CATGGA (65.48) & CACGGA (65.48) & 0\% \\
DYNLT1 & 59 & SRp55, SRp40 & ESE$^{1}$& TGAATC (78.47) & TGAATGC (79.10) & +0.81\% \\
& 60 & 9G8 & ESE$^{2}$ & GAATCC (60.07) & GAATGC (63.49) & +5.70\% \\
\hline
DYNLL2 & 2 & SF2/ASF (IgM-BRCA1) & ESE$^{1}$ & CTCCGGG (86.38) & CTCCGAG (72.69) & -15.85\% \\
& 2 & SF2/ASF, SF2/ASF (IgM-BRCA1) & ESE$^{1}$ & CTCCGGG (79.91) & CTCCGAG (72.69) & -9.03\% \\
& 4 & SF2/ASF (IgM-BRCA1) & ESE$^{1}$& CCGGGGT (73.00) & CCGAGGT (86.23) & 18.12\% \\
& 4 & SF2/ASF (IgM-BRCA1), SF2/ASF & ESE$^{1}$ & CCGGGGT (73.00) & CCGAGGT (82.94) & 13.61\% \\
& 6 & 9G8 & ESE$^{2}$ & & GAGGTG (60.67) & New site \\
& 6 & hnRNP A1 & ESS$^{4}$& & GAGGTG (74.05) & New site \\
\hline
PFN2 & 2068 & Tra2-$\beta$ & ESE$^{1}$ & AAAAT (81.02) & AAAAa & +16.19\% \\
& 2069 & Tra2-$\beta$ & ESE$^{1}$ & & AAAaa (94.14) & New site \\
& 2070 & Tra2-$\beta$ & ESE$^{1}$ & & AAAaaT (81.02) & New site \\
 & 2066 & & ESS$^{3}$ & & ACAAAAaa (38.13) & New site \\
 & 2067 & & ESS$^{3}$ & & CAAAAaaT (28.85) & New site \\
\hline

\end{tabular}
\begin{flushleft}
    $^{1}$ESE Finder matrices for SRp40, SC35, SF2/ASF and SRp55 proteins.
    $^{2}$ESE motifs from HSF.
    $^{3}$Predicted PESS Octamers from Zhang \& Chasin.
    $^{4}$hnRNP motif.
\end{flushleft}
\label{tab:spliceosome}
\end{table}

\end{document}
